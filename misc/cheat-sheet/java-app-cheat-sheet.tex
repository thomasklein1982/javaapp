\directlua{basispfad="../../../../Users/Thomas/Dropbox/Schule/tex/" context("\\input{"..basispfad.."basis}")}
\inputRelative{uebungen}
\inputRelative{info}
\setupfootertexts[]

\def\injava#1{\type[escape={[[,]]},option=java]{#1}}

\def\javaapp{{\externalfigure[../../res/icon][height=1.5ex]}}

\def\befehl#1#2{\item{\type[option=java]{#1}\\#2}}
\def\befehlJA#1#2{\item{\type[option=java]{#1} \javaapp\\#2}}

\def\nurCanvas{\color[red]{\tfx Klappt nur, wenn sich das Element in einem \mono{Canvas} befindet.}}

\setupbodyfont[10.5pt]

\setuppapersize[A4,landscape]

\setMargins{0.5cm}{-0.2cm}{30cm}

\setuphead
  [subject]
  [before={},
   after={\vskip-0.8\baselineskip},
   style=\tfb\bf]

\setuphead
  [subsubject]
  [before={},
   after={\vskip-0\baselineskip},
   style=\bf]

\starttext

\startcolumns[n=2]

\startsubject[title=UI]
Diese Methoden müssen von \mono{JComponent}-Elementen aufgerufen werden, z. B. \type[option=java]{label.setValue("Text");}.
\startsubsubject[title={Alle UI-Komponenten (\mono{JComponent})}]
\startitemize[4,nowhite]
\befehlJA{setValue(String)}{Ändert den Text/Wert des Elements.}
\befehlJA{getValue(): String}{Liefert den Text/Wert des Elements zurück.}
\befehlJA{show()/hide()}{Macht die Komponente sichtbar / unsichtbar.}
\befehlJA{setStyle(String css-eigenschaft, String wert)}{Legt eine CSS-Eigenschaft fest, z. B. \injava{setStyle("color","red")}}
\befehlJA{setStyle(String css-eigenschaft, String wert)}{Legt eine CSS-Eigenschaft fest, z. B. \injava{setStyle("color","red")}}
\befehl{setEnabled(boolean)}{Aktiviert (\mono{true}) bzw. deaktiviert (\mono{false}) die Komponente.}
\befehl{setActionCommand(String)}{Legt das \mono{ActionCommand} dieser Komponente fest.}
\befehl{getActionCommand(): String}{Gibt das \mono{ActionCommand} dieser Komponente zurück. Falls keines definiert wurde, wird der Text der Komponente zurückgegeben.}
\befehlJA{addEventListener(String event, (ev)->{Anweisungen})}{Legt fest, welche Anweisungen ausgeführt werden sollen, wenn ein bestimmtes Ereignis eintritt. Der Parameter \injava{ev} enthält Infos zu dem Event. Mögliche Events sind z.~B.:
\vskip-\baselineskip
\startitemize[1,packed][n=2,distance=6em]
\sym{\injava{"click"}} Komponente wird angeklickt.
\sym{\injava{"pointerover"}} Maus betritt die Komponente.
\sym{\injava{"pointerout"}} Maus bewegt sich von der Komponente weg.
\sym{\injava{"pointermove"}} Maus bewegt sich über der Komponente.
\sym{\injava{"pointerdown"}} Maustaste wird gedrückt über der Komponente.
\sym{\injava{"pointerup"}} Maustaste wird losgelassen über der Komponente.
\sym{\injava{"change"}} Der Wert der Komponente ändert sich.
\sym{\injava{"input"}} Es wird etwas in die Komponente eingegeben.
\stopitemize}
\column
\befehlJA{setX(double)/setY(double)}{Legt die $x$/$y$-Koordinate des Mittelpunkts fest. \nurCanvas}
\befehlJA{setWidth(double)/setHeight(double)}{Legt die Breite / Höhe fest. \nurCanvas}
\befehlJA{changeX(double)/...Y/...Width/...Height}{Ändert den entsprechenden Wert um die Angabe. \nurCanvas}
%\befehlJA{focus()}{Gibt der Komponente den Fokus.}
\stopitemize
\stopsubsubject


\startsubsubject[title={Container (\mono{JPanel}, \mono{JFrame} und \mono{Canvas})}]
\startitemize[4,nowhite]
\befehlJA{setLayout(String)}{Legt das \begriff{Layout} fest. Dieses bestimmt, wie die Kinder angeordnet werden. Syntax: \mono{Höhe Höhe ... Höhe / Breite Breite ... Breite}.\blank[small]

\startcombination[2*2]
{\startMPcode
setUnits(0.9cm,0.7cm);
draw rechteck((-1,0),7,3) tf withcolor white;
draw rechteck((0,0),6,3) tf;
for y=2:
	draw ((0,y)--(6,y)) tf dashed evenly;
endfor;
label.lft(textext("\mono{auto}"),(0,2.5) tf);
label.lft(textext("\mono{1fr}"),(0,1) tf);
for x=2,4:
	draw ((x,0)--(x,3)) tf dashed evenly;
endfor;
for x=1,3,5:
	label.bot(textext("\mono{1fr}"),(x,0) tf);
endfor;
\stopMPcode}{\tfx\injava{setLayout("auto 1fr / 1fr 1fr 1fr")}}
{\startMPcode
setUnits(0.9cm,0.7cm);
draw rechteck((-1,0),7,3) tf withcolor white;
draw rechteck((0,0),6,3) tf;
for y=2,0.5:
	draw ((0,y)--(6,y)) tf dashed evenly;
endfor;
label.lft(textext("\mono{1cm}"),(0,2.5) tf);
label.lft(textext("\mono{1fr}"),(0,2.75/2) tf);
label.lft(textext("\mono{0.5cm}"),(0,0.25) tf);
for x=1.8:
	draw ((x,0)--(x,3)) tf dashed evenly;
endfor;
label.bot(textext("\mono{30\%}"),(0.9,0) tf);
label.bot(textext("\mono{1fr}"),(1.8+4.2/2,0) tf);
\stopMPcode}{\tfx\injava{setLayout("1cm 1fr 0.5cm / 30[[\percent]] 1fr")}}
{\startMPcode
setUnits(0.9cm,0.7cm);
draw rechteck((-1,0),7,3) tf withcolor white;
draw rechteck((0,0),6,3) tf;
for x=1.5,3,4.5:
	draw ((x,0)--(x,3)) tf dashed evenly;
endfor;
\stopMPcode}{\tfx\injava{setLayout("4")}}
{\startMPcode
setUnits(0.9cm,0.7cm);
draw rechteck((-1,0),7,3) tf withcolor white;
draw rechteck((0,0),6,3) tf;
for y=1.5:
	draw ((0,y)--(6,y)) tf dashed evenly;
endfor;
for x=2,4:
	draw ((x,0)--(x,3)) tf dashed evenly;
endfor;
\stopMPcode}{\tfx\injava{setLayout("3/2")}}
\stopcombination
}
\befehl{add(JComponent)}{Fügt die Komponente hinzu.}
\befehl{remove(JComponent)}{Entfernt die Komponente.}
\befehlJA{getChild(int)}{Gibt die Kind-Komponente an der Stelle zurück.}
\befehlJA{getChildCount()}{Die Anzahl der Kind-Komponenten.}
\stopitemize
\stopsubsubject
\stopsubject

\column
\startsubject[title=Grafiken mit \mono{Canvas}]
Alle Methoden müssen von einem \mono{Canvas} ausgeführt werden.
\startsubsubject[title=Zeichnen-Befehle]
\startitemize[4,nowhite]
\befehlJA{clear()}{Löscht alle Zeichnungen.}
\befehlJA{drawLine(double x1,y1,x2,y2)}{Linie von $(\mono{x1}\mid \mono{y1})$ zu $(\mono{x2}\mid \mono{y2})$}
\befehlJA{drawCircle(double mx,my,r) / fillCircle(double mx,my,r)}{Kreis mit Mittelpunkt $(\mono{mx}\mid \mono{my})$ und Radius \mono{r}}
\befehlJA{drawRect(double mx,my,w,h) / fillRect(double mx,my,w,h)}{Rechteck mit Mittelpunkt $(\mono{mx}\mid \mono{my})$, Breite \mono{w}, Höhe \mono{h} }
\befehlJA{write(String t, double mx, double my, String align)}{Schreibt den Text \mono{t} an den Punkt $(\mono{mx}\mid \mono{my})$. Der optionale Parameter \mono{align} bestimmt die Ausrichtung:\vskip-\baselineskip
\startitemize[1,packed]
\item \mono{left/right/center}: Links/rechts oder in der Mitte.
\item \mono{top/bottom/middle}: Oben/unten oder in der Mitte.
\stopitemize
}
\befehlJA{drawImage(String url, double mx,my,w,h,winkel, boolean mirror)}{Zeichnet das Bild in das Rechteck mit Mittelpunkt $(\mono{mx}\mid \mono{my})$, Breite~\mono{w}, Höhe~\mono{h}, gedreht um \mono{winkel} und gespiegelt, wenn \injava{mirror==true}.  }

\stopitemize
\stopsubsubject

\startsubsubject[title=Canvas-Einstellungen]
\startitemize[4,nowhite]
\befehlJA{setAxisX(double min, double max)}{Legt fest, bei welchem $x$-Wert der Canvas startet und bis zu welchem $x$-Wert er geht.}
\befehlJA{setAxisY(double min, double max)}{Legt fest, bei welchem $y$-Wert der Canvas startet und bis zu welchem $y$-Wert er geht.}
\befehlJA{setSizePolicy(String)}{Legt fest, ob sich der Canvas ausdehnen soll  (\injava{"stretch"}) oder ob er eingepasst werden soll (\injava{"fit"}).\vskip-0.5\baselineskip

\midaligned{\startcombination[2*1]
{\startMPcode
setUnits(1cm,1cm);
draw rechteck((0,0),3,2) tf;
path p;
p:=rechteck((0.5,0),2,2) tf;
fill p withcolor red;
draw p dashed evenly;
label.bot(textext("\injava{fit}"),(1.5,0) tf);
\stopMPcode
}{}
{\startMPcode
setUnits(1cm,1cm);
draw rechteck((0,0),3,2) tf;
path p;
p:=rechteck((0,0),3,2) tf;
fill p withcolor red;
draw p dashed evenly;
label.bot(textext("\injava{stretch}"),(1.5,0) tf);
\stopMPcode}{}
\stopcombination}

}
\column

\startsubsubject[title=Zeichnen-Einstellungen]
Jeder dieser Befehle beeinflusst alle {\bf nachfolgenden Zeichnen-Befehle}.
\vskip-\baselineskip
\startitemize[4,packed]
\befehlJA{setColor(String)}{Legt die Farbe fest, z.~B. \injava{setColor("red");} oder \injava{setColor("[[\#]]FF69B4");}}
\befehlJA{setOpacity(double)}{Legt die Transparenz fest von $0.0$ (unsichtbar) über $0.5$ (halb durchscheinend) bis $1.0$ (keine Transparenz).}
\befehlJA{setLinewidth(double)}{Legt Liniendicke fest.}
\befehlJA{setFontsize(double)}{Legt die Schriftgröße fest.}
\befehlJA{setFont(String)}{Legt die Schriftart fest, z.~B. \injava{setFont("Arial");}}
\befehlJA{setMirrored(boolean)}{\injava{setMirrored(true);} $\Rightarrow$ alle Zeichnungen spiegeln.}
\befehlJA{rotate(double w, double mx, double my)}{Dreht die Zeichnung um den Winkel \mono{w} um den Mittelpunkt $(\mono{mx}\mid\mono{my})$.}
\befehlJA{scale(double sx, double sy, double mx, double my)}{Vergrößert/verkleinert die Zeichung vom Mittelpunkt $(\mono{mx}\mid\mono{my})$ aus mit dem Vergrößerungsfaktor \mono{sx} in $x$-Richtung und \mono{sy} in $y$-Richtung.}
\befehlJA{translate(double dx, double dy)}{Verschiebt die Zeichnung um \mono{dx} in $x$-Richtung und um \mono{dy} in $y$-Richtung.}
\befehlJA{setTransform(double m00, double m10, double m01, double m11, double m02, double m12)}{Legt die Transformationsmatrix fest auf 

\matrix{m_{00};m_{01};m_{02}|m_{10};m_{11};m_{12}|0;0;1}
}
\befehlJA{reset()}{Setzt alle Eigenschaften auf die Ursprungswerte zurück.}
\befehlJA{save()}{Speichert die aktuellen Eigenschaften des Canvas.}
\befehlJA{restore()}{Stellt die zuletzt gespeicherten Eigenschaften des Canvas wieder her.}

\stopitemize 
\stopsubsubject

\stopsubject
\column
\startsubject[title={Assets}]
\startitemize[1,packed]
\item Assets sind statische Dateien wie Bilder, Sounds, Videos usw. 
\item Assets werden in JavaApp hochgeladen (Projekt \rightarrow\ Assets). 
\item Um ein Asset in HTML oder CSS zu verwenden, muss man \mono{asset(name)} schreiben, wobei \mono{name} der Name des Assets in JavaApp ist: 

\midaligned{\type{<img src="asset(bild)"/>}}
\stopitemize
\stopsubject

\startsubject[title=Sound]
Erst MP3-Datei als Asset hochladen und Namen geben.
\vskip-\baselineskip
\startitemize[4,packed]
\befehlJA{new Sound(String asset)}{Erzeugt ein neues \mono{Sound}-Objekt. Der Parameter ist der Name der entsprechenden Datei, die als Asset hochgeladen wurde.}
\befehlJA{play(boolean)}{Startet die Wiedergabe (ggf. neu). Falls der Parameter \mono{true} ist, in Endlosschleife.}
\befehlJA{pause()}{Pausiert die Wiedergabe.}
\befehlJA{resume()}{Beendet die Pausierung.}
\befehlJA{stop()}{Beendet die Wiedergabe und setzt den Sound auf Anfang zurück.}
\befehlJA{setSource(String)}{Legt fest, welche Datei abgespielt werden soll.}
\befehlJA{getDuration(): int}{Liefert die Wiedergabezeit des Sounds in Millisekunden zurück.}
\befehlJA{getCurrentTime(): int}{Liefert die Wiedergabeposition zurück (in Millisekunden).}
\befehlJA{isEnded(): boolean}{Liefert \mono{true}, falls die Wiedergabe beendet ist, ansonsten \mono{false}.}
\befehlJA{static beep(int frequency, double volume, int duration)}{Spielt einen Ton ab. Bsp: \injava{Sound.beep(440,1,1000);}}
\stopitemize
\stopsubject

\startbuffer[addbuttonlistener]
\startjava[numbering=no]
gamepad.addButtonListener("B", (ev)->{
	mario.springen();
});
\stopjava
\stopbuffer


\column
\startsubject[title=JImage]
UI-Komponente, die ein Bild darstellt. Erst PNG oder JPG als Asset hochladen und Namen geben.
\vskip-\baselineskip
\startitemize[4,nowhite]
\befehlJA{new JImage(String asset)}{Erzeugt ein neues \mono{JImage}-Objekt.}
\stopitemize
\stopsubject

\startsubject[title=Gamepad]
Jedes Gamepad-Objekt erzeugt ein On-Screen-Gamepad, das automatisch mit einem physischen Gamepad verbunden wird, falls ein solches Gamepad an das Gerät angeschlossen wird.
\vskip-\baselineskip
\startitemize[4,nowhite]
\befehlJA{new Gamepad()}{Erzeugt ein neues Gamepad und zeigt es auf dem Bildschirm an.}
\befehl{setEventListener(String button, String event, (ev)->{Anweisungen})}{
Legt fest, was passieren soll, wenn ein Button gedrückt bzw. losgelassen wird.
\vskip-\baselineskip
\startitemize[1,packed]
\item \mono{button}: \injava{"A"}, \injava{"B"}, \injava{"X"}, \injava{"Y"}, \injava{"LEFT"}, \injava{"RIGHT"}, \injava{"UP"} oder \injava{"DOWN"}.
\item \mono{event}: \injava{"press"} oder \injava{"release"} oder \injava{"click"}
\stopitemize

}
\befehlJA{setPosition(String left, String bottom)}{Legt den Abstand zum linken unteren Bildschirmrand fest.\\
Z.~B. \injava{setPosition("2cm","3cm")}}
\befehlJA{setWidth(String width)}{Legt die Breite des Gamepads inklusive Padding fest.\\
Z.~B. \injava{setWidth("50[[\percent]]")}}
\befehlJA{setPadding(String p)}{Legt fest, wie viel Abstand das Steuerkreuz und die ActionButtons vom Rand haben sollen. Z.~B. \injava{setPadding("0.5cm")}}
\befehlJA{getDirection(): String}{Liefert einen String zurück, der die Richtung beschreibt, in die das Steuerkreuz gerade zeigt: \injava{"n"}, \injava{"s"}, \injava{"w"}, \injava{"e"}, \injava{"nw"}, \injava{"ne"}, \injava{"sw"} oder \injava{"se"}.}
\befehlJA{isLeftPressed()/isRightPressed()/isUpPressed()/isDownPressed(): boolean}{Liefert \mono{true} zurück, wenn der entsprechende Richtungsbutton gedrückt ist, sonst \mono{false}.}
\stopitemize
\stopsubject

\column
\startsubject[title=Konsole]
\startitemize[4,nowhite]
\befehl{System.out.println(String)}{Gibt eine Zeile in der Konsole aus.}
\befehl{System.in.read()}{Wartet auf einen Tastendruck/Klick des Users.}
\befehl{System.console().readLine(): String}{Wartet darauf, dass der User etwas eingibt. Gibt die Eingabe zurück.}
\befehlJA{System.clear()}{Löscht die gesamte Konsole.}
\stopitemize
\stopsubject

\stopcolumns

\stoptext
